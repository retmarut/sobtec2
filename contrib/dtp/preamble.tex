% Lot's of language independant stuff for
% consistancy and fancyness.

\usepackage{lmodern}

% insert custom language by passing it through the languagename variable
\usepackage[\languagename]{babel}

% testing
%\usepackage{showframe}

% page layout headers&footers, pagenumbering layout
\usepackage[headheight=30pt,height=6.5in]{geometry}

%\usepackage[headheight=20pt,height=240mm,width=160mm]{geometry}
% pages of 240x160 (bigger than A5)
%\usepackage[headheight=20pt,height=180mm]{geometry}

% headers and footers
\usepackage{fancyhdr}
\fancyhf{} % clear 
\renewcommand{\headrulewidth}{0pt}
\renewcommand{\sectionmark}[1]{\markboth{#1}{}} % set the \leftmark
%\fancyhead[L]{\leftmark} % 1. sectionname
\fancyfoot[LE,RO]{\thepage}
\pagestyle{fancy}
\fancypagestyle{plain}{\pagestyle{fancy}}

% for tables 
\usepackage{longtable}

% there are inline images per article/section, so
\usepackage{graphicx}
%https://tex.stackexchange.com/questions/39147/scale-image-to-page-width
%http://texdoc.net/texmf-dist/doc/latex/graphics/grfguide.pdf
\setkeys{Gin}{width=0.9\textwidth}
% disable default image captions everywhere
\usepackage[labelformat=empty]{caption}

% Custom license package on $TEXFM with dutch translation was used 
% For generic info see https://github.com/ypid/latex-packages/tree/master/doclicense
\usepackage[ type={CC}, modifier={by-nc-sa}, version={3.0},
	imagewidth={8em}]{doclicense}

% needed to render UTF8 and UTF16 chars
\usepackage{ifxetex,ifluatex}
\ifnum 0\ifxetex 1\fi\ifluatex 1\fi=0 % if pdftex
 \usepackage[T1]{fontenc}
 \usepackage[utf8]{inputenc}
\else % if luatex or xelatex
 \ifxetex
  \usepackage{mathspec}
 \else
  \usepackage{fontspec}
 \fi
 \defaultfontfeatures{Ligatures=TeX,Scale=MatchLowercase}
\fi

% use upquote if available, for straight quotes in verbatim environments
\IfFileExists{upquote.sty}{\usepackage{upquote}}{}

% use microtype if available
% http://www.khirevich.com/latex/microtype/
\IfFileExists{microtype.sty}{
	\usepackage{microtype}
	\UseMicrotypeSet[protrusion]{basicmath} % disable protrusion for tt fonts
}{}

% Documents frequently uses URL in footnotes
%\usepackage[hyphenbreaks]{breakurl}
\usepackage[hyphens]{url}
\usepackage{hyperref}
\hypersetup{unicode=false, pdfborder={0 0 0}}
\urlstyle{same}  % don't use monospace font for urls

\usepackage[parfill]{parskip}
\setlength{\emergencystretch}{3em}  % prevent overfull lines
\providecommand{\tightlist}{%
  \setlength{\itemsep}{0pt}
  \setlength{\parskip}{0pt}}

% Redefines (sub)paragraphs to behave more like sections
\ifx\paragraph\undefined\else
 \let\oldparagraph\paragraph
 \renewcommand{\paragraph}[1]{\oldparagraph{#1}\mbox{}}
\fi
\ifx\subparagraph\undefined\else
 \let\oldsubparagraph\subparagraph
 \renewcommand{\subparagraph}[1]{\oldsubparagraph{#1}\mbox{}}
\fi

% no section and sub-section numbering anywhere
\setcounter{secnumdepth}{0}
% only show part,chapters,sections in toc. No subsections
\setcounter{tocdepth}{1} 
%always start sections on odd pages (right)
\let\oldsection\section
\def\section{\cleardoublepage\oldsection}

